%!TEX root = ../dissertation.tex

En 2014, l’OMS estimait à 8000, le nombre de décès par an au Cameroun \footnote{\href{https://www.medcamer.org/wp-content/uploads/2018/03/CISa-2018_Presentation_Dr-Bassong.pdf}{https://www.medcamer.org/wp-content/uploads/2018/03/CISa-2018_Presentation_Dr-Bassong.pdf}}, suite à des erreurs médicales, erreurs qui sont encore plus accentuées avec le manque d'expérience de nos jeunes médecins fraîchement sortis de l'école, qui cherchent de l'expérience, mais qui souvent l'acquièrent en faisant au passage des dégâts souvent irréversibles. Parmi l'expérience à acquérir par ces jeunes médecins, on retrouve la pose de diagnostic, qui est un exercice assez complexe qui combine des bagages à la fois de savoir et de savoir-faire. Ce document propose une modélisation d’un STI qui pourra être pour les étudiants médecins et jeunes médecins, un cadre d’expérimentation à la pose de diagnostic. Le décor a été posé en présentant les concepts généraux et une revue de la littérature au sujet des STI et du diagnostic médical. Puis, une architecture globale d'un STI a été décrite. Il y aura un modèle apprenant qui nous permettra de décrire et de suivre la trace de l’évolution de l’apprentissage de l’apprenant, son niveau, ses lacunes, ses idées fausse; un modèle tuteur qui permet de décrire notre stratégie pédagogique : l’entraînement; un modèle expert qui décrit une représentation des connaissances à transmettre à l’apprenant notamment l’identification de la maladie ( sous forme de relation ”si tels symptômes, alors telle maladie”) et la procédure de pose de diagnostic ( en utilisant une base de faits et de règles et les réseaux bayésiens pour l’inférence); un patient virtuel, modélisé comme un agent émotif et réactif, pour simuler une maladie et répondre aux questions posées par l’apprenant; et, enfin l’interface utilisateur par lequel l’apprenant passera pour interagir avec notre STI. La conception de l’interface utilisateur a suivi la méthodologie de la conception centrée sur utilisateur présentée dans la norme ISO 13407. Il y a donc eu une analyse permettant identification et la caractérisation de la population cible et son environnement; puis, une conception permettant la spécification des exigences des utilisateurs. Ensuite, les outils, les algorithmes et les interfaces qui seront utilisés pour l'implémentation de chaque composant du STI ont été présentés. A ce propos, dans le module tuteur, JESS sera utilisé pour implémenter la base de règles. La base de cas du module Expert sera enregistrée dans une base de données NoSQL, MongoDB en occurrence. Le réseau bayésien sera mis en place avec l’outil pyArgum.L’interface utilisateur a été prototypée sur l’approche High Fidelity avec l’outil Figma.

\textbf{Mots clés}: systèmes tutoriels intelligents, pose de diagnostic.