%!TEX root = ../dissertation.tex

En 2014, l’OMS estimait à 8000, le nombre de décès par an au Cameroun, suite à des erreurs médicales, erreurs qui sont encore plus accentuées avec le manque d'expérience de nos jeunes médecins fraîchement sortis de l'école, qui cherchent de l'expérience, mais qui souvent l'acquièrent en faisant au passage des dégâts souvent irréversibles. Parmi l'expérience à acquérir par ces jeunes médecins, on retrouve la pose de diagnostic, qui est un exercice assez complexe et combine des bagages à la fois de savoir et de savoir-faire.

C'est dans cette logique que s'inscrit notre solution, nous proposons d'utiliser tout le potentiel des STIs, qui est une nouveau paradigme d'apprentissage en pleine expansion dans le domaine de l'intelligence artificielle pour l'éducation, pour proposer un cadre adéquat aux jeunes médecins pour s'exercer à l'art de la pose de diagnostic. 

Ce document présente tout d'abord le contexte, la problématique et les objectifs du projet. Ensuite nous faisons une étude détaillée des concepts sous-jacents aux STI. Notre document présente également notre démarche méthodologique, les choix techniques d'implémentation avec les raisons qui les ont motivés, et une méthode d'évaluation de notre solution.

\textbf{Mots clés}: systèmes tutoriels intelligents, pose de diagnostic.