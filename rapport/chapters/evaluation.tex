%!TEX root = ../dissertation.tex

\chapter{Discussion}
\label{chp:disc}


\section{Points forts}
Les points forts de notre solution: 

\begin{itemize}
    \item Adaptation du contenu (des CAS) en fonction du niveau réel de l'apprenant médecin.
    \item Utilisation de plusieurs types de feedback qui fournissent à l'apprenant un retour sur sa performance pendant la séance, et sur les moyens de l'améliorer.
    \item Prise en compte de l'état cognitif et affectif pour améliorer la stratégie pédagogique.
    \item Modélisation des procédures de pose de diagnostic pour facilement évaluer la procédure de l'apprenant.
    \item Prise en compte des examens dans notre système, pour rendre la pose de diagnostic encore plus réelle et variée.
    \item Utilisation des discussions par voix et par textes pour rendre l'expérience encore plus réaliste. 
\end{itemize}

\section{Limites}
Les limites de notre solution:

\begin{itemize}
    \item Le suivi d'un patient n'est pas pris en compte (les CAS liés).
    \item La simulation des expressions du patient virtuel n'est pas pris en compte.
\end{itemize}

\section{Perspectives}

Les perspectives de notre solution:

\begin{itemize}
    \item On pourrait ajouter la prescription des médicaments dans la procédure de la pose de diagnostic
    \item On pourrait utiliser le modèle inférentiel pour rendre la stratégie pédagogique encore plus efficace.
    \item On pourrait ajouter l'évaluation des expressions non verbales du patient par l'apprenant médecin.
\end{itemize}