%!TEX root = ../dissertation.tex

\chapter{Conclusion}
\label{chp:conclusion}

En somme, il était question tout au long de ce document de proposer aux étudiant médecins et jeunes médecin, un cadre d'expérimentation à la pose de diagnostic. Pour y parvenir, nous avons proposé une modélisation d'un STI. Tout d'abord, nous avons posé le décor en présentant les concepts généraux et une revue de la littérature au sujet des STI et du diagnostic médical. Puis, nous avons détaillé l'architecture globale de notre solution. Nous aurons un modèle apprenant qui nous permettra de d'écrire et de suivre la trace de l’évolution de l’apprentissage de l’apprenant, son niveau, ses lacunes, ses idées fausse; un modèle tuteur qui permet de décrire notre stratégie pédagogique : l'entraînement ; un modèle expert qui décrit une représentation des connaissances à transmettre à l'apprenant notamment l'identification de la maladie ( sous forme de relation "si tels symptômes, alors tel maladie") et la procédure de pose de diagnostic ( en utilisant une base de faits et de règles et les réseaux bayésiens pour l'inférence); un patient virtuel pour simuler une maladie et répondre aux questions posées par l'apprenant que nous modélisons comme un agent émotif et réactif; et, enfin l'interface utilisateur par lequel l'apprenant passera pour interagir avec notre STI. La conception de l'interface utilisateur a suivi la méthodologie de la conception centrée sur utilisateur présentée dans la norme ISO 13407. Il y a donc eu une analyse où nous avons identifier et caractériser notre population cible et son environnement; puis, une conception où nous avons spécifier leur exigences des utilisateurs. Ensuite, nous avons touché l'aspect pratique de tout ceci dans la partie l'implémentation. Nous y avons présenté les outils, les algorithmes et les interfaces qui seront utilisés dans chaque composant de notre STI. A ce propos, dans le module tuteur, nous utiliserons JESS pour implémenter la base de règles. L'interface utilisateur a été prototypé sur l'approche High Fidelity avec l'outil Figma. Le module Expert , .... AHHH SHITTTT !