%!TEX root = ../dissertation.tex

\chapter{Introduction}
\label{chp:intro}

%Random citation \cite{goodfellow2016deep}.

\section{Contexte et problématique}

En 2014, l'OMS estimait à $8000$, le nombre de décès par an au Cameroun, suite à des erreurs médicales. Ceci dû à plusieurs causes, notamment une mauvaise communication, un transfert partiel des informations en cas de référence, des problèmes humains, de mauvaises politiques de santé, un transfert des connaissances pas assez important etc ...  Par ailleurs nos jeunes médecins fraîchement sortis de l'école qui s'en vont dans les différentes structures sanitaires afin d'exercer, manque cruellement d'expérience et donc, sont plus enclin à faire des erreurs médicales. Ils acquièrent effectivement de l'expérience via ce processus, mais en faisant au passage des dégâts souvent irréversibles.


De cette situation ce dégage donc un problème fondamental, comment transmettre, efficacement, de l'expérience aux jeunes médecins ? 

\newpage

\section{Objectif}
Pour répondre à ce problème nous nous proposons de mettre sur pied un Système Tutoriel Intelligent (STI) pour l'apprentissage de la pose d'un diagnostic médical. Il s'agit là, d'une solution idéale, car les STIs sont dotés d 'environnements permettant d 'offrir un apprentissage individualisé, adaptatif et de qualité pour nos jeunes médecins. Pour atteindre cet objectif, nous nous proposons de :

\begin{itemize}
    \item Concevoir une architecture du système basé sur l'architecture classique des STIs.
    \item Éliciter des connaissances brutes exprimées par les experts (il s'agit entre autre d'identifier et de faire ressortir les concepts importants du domaine et les relations significatives entre ces concepts).
    \item Construire un modèle opérationnel des concepts et des relations identifiés à
l'étape précédente (il s'agit de la formalisation des connaissances).
    \item Construire un modèle de représentation des connaissances et des performances de l'apprenant (le jeune médecin) comprenant des outils de diagnostic, ayant pour objectif d'identifier les erreurs et les fausses idées et offrant des mécanismes pour aider à les corriger.
    \item Établir des règles tutorielles basées sur les stratégies de tutorat dont l'efficacité a été prouvée et dont l'application est très répandue dans les environnements d'apprentissage;
\end{itemize}


\newpage

\section{Plan du rapport}
Ce rapport de recherche contient six chapitres au total. Ces chapitres peuvent être regroupés en cinq grandes parties:\\
La première partie est celle de l'introduction avec comme sous-sections le contexte et la problématique, l'objectif et le plan du mémoire comme dernier point.\\
La deuxième partie est celle de l'état de l'art. Ici il s'agit de rechercher toutes les informations existantes concernant ce domaine et à en faire une synthèse. Pour cela nous avons procédé en trois étapes, la première partie définir ce qu'on appelle diagnostic médicale, ensuite présenter l'historique et définir les systèmes tutoriels intelligents et enfin présenter les composantes d'un STI.  \\
La troisième partie est celle de la présentation de l'architecture, ici nous avons procédé en présentant les différentes composantes du système et ensuite présenter l'architecture fonctionnelle du système.\\
La quatrième partie regroupe la partie Implémentation et évaluation. Ici il s'agissait pour nous de présenter les techniques, langages et les outils utilisés pour mettre en oeuvre notre système. \\
La cinquième et dernière partie contient uniquement le chapitre 5. Elle présente essentiellement la conclusion.\\